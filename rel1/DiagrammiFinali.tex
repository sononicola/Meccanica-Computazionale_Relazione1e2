\begin{figure}[htb]\centering
\subfloat[][\emph{Momento flettente}]
{
    \begin{tikzpicture}
    \scaling{2.5};
    \begin{scope}[color=lightgray]
        \point{a}{0}{0};
        %\point{b}{.5}{0};
        \point{c}{1}{0};
        \point{d}{2}{.5};
        \point{e}{2}{1};
        \point{f}{1}{1};
        \beam{2}{a}{c}[0][1];
        \beam{2}{c}{d}[1][1];
        \beam{2}{d}{e}[1][1];
        \beam{2}{e}{f}[1][1];
        \beam{2}{f}{c}[1][1];
    \end{scope}
%Diagramma momento:
    \internalforces{a}{c}{.38}{-.6};
    \internalforces{c}{f}{-.6}{0};
    \internalforces{f}{e}{0}{-.3}[-0.1];
    \internalforces{e}{d}{-.3}{-.6};
    \internalforces{d}{c}{-.6}{0};
    \end{tikzpicture}
}
\subfloat[][\emph{Sforzo tagliante}]
{
    \begin{tikzpicture}
    \scaling{2.5};
    \begin{scope}[color=lightgray]
        \point{a}{0}{0};
        %\point{b}{.5}{0};
        \point{c}{1}{0};
        \point{d}{2}{.5};
        \point{e}{2}{1};
        \point{f}{1}{1};
        \beam{2}{a}{c}[0][1];
        \beam{2}{c}{d}[1][1];
        \beam{2}{d}{e}[1][1];
        \beam{2}{e}{f}[1][1];
        \beam{2}{f}{c}[1][1];
    \end{scope}
%Diagramma taglio:
    \internalforces{a}{c}{-.8}{-.8}[0][blue];
    \internalforces{c}{f}{.49}{.49}[0][blue];
    \internalforces{f}{e}{0}{-.45}[0][blue];
    \internalforces{e}{d}{-.55}{-.55}[0][blue];
    \internalforces{d}{c}{.49}{.49}[0][blue];
    \end{tikzpicture}
}
\subfloat[][\emph{Sforzo normale}]
{
    \begin{tikzpicture}
    \scaling{2.5};
    \begin{scope}[color=lightgray]
        \point{a}{0}{0};
        %\point{b}{.5}{0};
        \point{c}{1}{0};
        \point{d}{2}{.5};
        \point{e}{2}{1};
        \point{f}{1}{1};
        \beam{2}{a}{c}[0][1];
        \beam{2}{c}{d}[1][1];
        \beam{2}{d}{e}[1][1];
        \beam{2}{e}{f}[1][1];
        \beam{2}{f}{c}[1][1];
    \end{scope}
%Diagramma normale
    \internalforces{f}{e}{-.32}{-.32}[0][orange];
    \internalforces{e}{d}{.28}{.28}[0][orange];
    \internalforces{d}{c}{.49}{.49}[0][orange]
    \end{tikzpicture}
}
\subfloat[][\emph{Deformazione}]
{
    \begin{tikzpicture}
    \scaling{2.5};
%Indeformata
    \begin{scope}[color=lightgray]
        \point{a}{0}{0};
        %\point{b}{.5}{0};
        \point{c}{1}{0};
        \point{d}{2}{.5};
        \point{e}{2}{1};
        \point{f}{1}{1};
        \beam{2}{a}{c}[0][1];
        \beam{2}{c}{d}[1][1];
        \beam{2}{d}{e}[1][1];
        \beam{2}{e}{f}[1][1];
        \beam{2}{f}{c}[1][1];
    \end{scope}
%Deformata
    \begin{scope}[color=teal]
        \point{n}{1}{-0.01};
        \point{h}{2.03}{.43};
        \point{i}{2.02}{0.93};
        \point{l}{1.01}{.98};
        \point{m}{1.01}{.88} ;
        \beam{2}{a}{n}[0][1];
        \beam{2}{c}{h}[1][1];
        \beam{2}{h}{i}[1][1];
        \beam{2}{i}{m}[1][1];
        \beam{2}{l}{n}[1][1];
    \end{scope}
    \end{tikzpicture}
}    
\caption{Rappresentazione dei diagrammi delle sollecitazioni agenti sulla struttura e la relativa deformata}
\label{fig:DiagrammiFinali}
\end{figure}