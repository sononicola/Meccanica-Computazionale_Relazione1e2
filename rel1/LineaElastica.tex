\pagebreak
\section{Analisi tramite linea elastica}
Come già anticipato per calcolare la soluzione analitica è stato utilizzato il metodo della linea elastica tramite l'ausilio del programma Mathematica.
Dopo aver fissato un sistema di riferimento globale e dei sistemi di riferimento locali per ogni elemento, si è proceduto con la definizione delle equazioni differenziali. 

La prima equazione differenziale di quarto grado omogenea o non omogenea di ogni elemento definisce la deformazione a flessione, mentre la seconda equazione differenziale del secondo ordine omogenea definisce la deformazione assiale della trave.

{\footnotesize{
\begin{align*}
    EI\frac{d^4\,v_1(x)}{d\,x^4} = 0 &\qquad
    EA\frac{d^2\,u_1(x)}{d\,x^2} = 0 \\
    EI\frac{d^4\,v_2(x)}{d\,x^4} = 0 &\qquad
    EA\frac{d^2\,u_2(x)}{d\,x^2} = 0 \\
    EI\frac{d^4\,v_3(x)}{d\,x^4} = q &\qquad
    EA\frac{d^2\,u_3(x)}{d\,x^2} = 0 \\
    EI\frac{d^4\,v_4(x)}{d\,x^4} = 0 &\qquad
    EA\frac{d^2\,u_4(x)}{d\,x^2} = 0 \\
    EI\frac{d^4\,v_5(x)}{d\,x^4} = 0 &\qquad
    EA\frac{d^2\,u_5(x)}{d\,x^2} = 0 
\end{align*}}}
In totale perciò si hanno 10 equazioni differenziali per un totale di 30 costanti di integrazione da definire e quindi 30 condizioni al contorno per risolvere il sistema.
Nella tabella \ref{tab:EqContornoLineaElastica} sono state riportate per ogni nodo le equazioni di congruenza tra gli spostamenti e le equazioni di equilibrio per le sollecitazioni.
\begin{table}[htb]
\footnotesize
\caption{Elenco delle condizioni al contorno adottate nei nodi della struttura}
\label{tab:EqContornoLineaElastica}
\centering
\[
\begin{array}{ccc}
\toprule	
\text{Nodo} & \text{Equazioni di congruenza} & \text{Equazioni di equilibrio} \\\midrule
\multirow{3}{*}{1} & u_1(0)=0 & \\
 & v_1(0)=0 & \\
 & v_1^\prime(0)=0 & \\\midrule
\multirow{5}{*}{2} & u_1\left(L\right)=-u_5\left(\sqrt{\frac{5}{4}}L\right)\frac{2}{\sqrt{5}} - v_5\left(\sqrt{\frac{5}{4}}L\right)\frac{1}{\sqrt{5}} & M_1(L) - M_2(0) =0\\
 & v_1\left(L\right)=u_5\left(\sqrt{\frac{5}{4}}L\right)\frac{1}{\sqrt{5}} - v_5\left(\sqrt{\frac{5}{4}}L\right)\frac{2}{\sqrt{5}} & N_2(0) - T_1(L) + T_5\left(\sqrt{\frac{5}{4}}L\right)\frac{2}{\sqrt{5}} - N_5\left(\sqrt{\frac{5}{4}}L\right)\frac{1}{\sqrt{5}}=0\\
 & u_1(L) = v_2(0) & T_2(0) - N_1(L) + N_5\left(\sqrt{\frac{5}{4}}L\right)\frac{2}{\sqrt{5}} + N_5\left(\sqrt{\frac{5}{4}}L\right)\frac{1}{\sqrt{5}}=0\\
 & v_1(L) = -u_2(0) & M_5\left(\sqrt{\frac{5}{4}}L\right)=0\\
 & v_1^\prime(L) = v_2^\prime(0) & \\\midrule
\multirow{5}{*}{3} & v_2(L) = u_3(0) & M_2(L)=0\\
 & & N_2(L)=0 \\
 & & M_3(0)=0 \\
 & & T_3(0)=0 \\
 & & N_3(0) - T_2(L) = k\, v_2(L)\\\midrule
\multirow{3}{*}{4} & u_3(L) = -v_4(0) & N_3(0)+T_4\left(\sqrt{\frac{5}{4}}L\right)=0 \\
 & v_3(L) = u_4(0) & N_4(0)-T_3\left(\sqrt{\frac{5}{4}}L\right)=0 \\
 & v_3^\prime(L) = v_4^\prime(0) & M_3\left(\sqrt{\frac{5}{4}}L\right)-M_4(0)=0 \\\midrule
\multirow{3}{*}{5} & v_4(L/2) = u_5(0)\frac{2}{\sqrt{5}} + v_5(0)\frac{1}{\sqrt{5}} & M_4(L/2) - M_5(0)=0\\
 & u_4(L/2) = u_5(0)\frac{1}{\sqrt{5}} - v_5(0)\frac{2}{\sqrt{5}} & T_4(L/2)-T_5(0)\frac{1}{\sqrt{5}} - N_5(0)\frac{2}{\sqrt{5}}=0\\
 & v_4^\prime(L/2) = v_5^\prime(0) & N_4(L/2)+T_5(0)\frac{2}{\sqrt{5}}-N_5(0)\frac{1}{\sqrt{5}}-F=0\\
\bottomrule
\end{array}
\]
\end{table}

Si è quindi risolto il sistema di equazioni differenziali tramite l'ausilio di Mathematica, trovando le funzioni degli spostamenti al variare di $x$:

{\footnotesize{
\begin{align*}
     u_1(x) &= -2.4087\times10^{-7}x\\
     v_1(x) &= -0.000415408 x^3 + 0.00131629 x^2 \\
     u_2(x) &= -0.00063059\\
     v_2(x) &= -0.00026418 x^3 +0.00237762 x^2-0.00331828 x-7.2234\times10^{-7}\\
     u_3(x) &= 2.63158 \times10^{-6}x +0.00431015\\
     v_3(x) &= 0.0000207704 x^4 -00734869x + 0.0379719\\
     u_4(x) &= -2.25564\times10^{-6} x+0.0176082\\
     v_4(x) &= 0.000290786 x^3+0.0011216 x^2-0.00510549 x-0.00431804\\
     u_5(x) &= -4.03501\times10^{-6} x+0.000296188\\
     v_5(x) &= -0.000241509 x^3+0.00243014 x^2 +0.000222117 x-0.0195347\\
\end{align*}}}
A cui poi applicando le formule di legame tra gli sforzi e sostituendo i valori di $x$, si è pervenuti ai valori di spostamento nel nodo richiesto ed ai valori di sforzo per il primo nodo. 
\begin{align*}
    M(x) &= -EI \left[v^{\prime\prime}(x) + \chi_{th}\right] \\
    T(x) &= M^\prime(x) \\
    N(x) &= EA \, u^\prime (x)
\end{align*}
\begin{align*}
    u_2 &= \SI{-7.22e-7}{\meter}\\
    v_2 &= \SI{6.31e-4}{\meter}\\
    M_1 &= \SI{-7.362e3}{\newton\meter}\\
    T_1 &= \SI{1.500e3}{\newton}\\
    N_1 &= \SI{-96.07}{\newton}\\
\end{align*}

