\section{Analisi tramite FEM}
%Input abaqus
\begin{lstlisting} 
*HEADING
ESERCITAZIONE DI MECCANICA COMPUTAZIONALE
Unita' di misura SI (Pa, N, m, ...)
1-asse orizzontale, 2-asse verticale
*PREPRINT, ECHO=YES, MODEL=YES, HISTORY=YES

*NODE, NSET=NALL
10, 0, 0, 0
20, 3, 0, 0
21, 3, 0, 0
30, 3, 3, 0
31, 3, 3, 0
40, 6, 3, 0
50, 6, 1.5, 0
*ELEMENT, TYPE=B23, ELSET=alfavero
2, 20, 30
3, 31, 40
4, 40, 50
5, 50, 21
*ELEMENT, TYPE=B23, ELSET=alfazero
1, 10, 20
*ELEMENT, TYPE=SPRING1, ELSET=MOLLA
6,30
*SPRING, ELSET=MOLLA
1
100
*MPC
PIN, 20, 21
*EQUATION 
2 
30, 1, 1.0, 31, 1, -1.0
*INITIAL CONDITIONS, TYPE=TEMPERATURE
NALL,0
*BEAM GENERAL SECTION, ELSET=alfazero, SECTION=GENERAL
0.004656,0.0000292335,0, 0.0000292335,0
0,0,-1
200.E9,0.3,0.000012
*BEAM GENERAL SECTION, ELSET=alfavero, SECTION=GENERAL
0.004656,0.0000292335,0, 0.0000292335,0
0,0,-1
200.E9,0.3,0
*STEP, PERTURBATION
*STATIC
*BOUNDARY
10,ENCASTRE
*CLOAD
50,2,-2.E3
*DLOAD
3,PY,-6.E3
*TEMPERATURE, OP=MOD
10,0,100,0
20,0,100,0
*NODE PRINT
U,
RF,
*EL PRINT, POSITION=NODES, SUMMARY=YES
SF,
SM,
*END STEP
\end{lstlisting}