\section{Analisi tramite FEM}
Per la risoluzione con questa metodologia è stato utilizzato il software Abaqus. 
Sono stati scelti degli elementi finiti di tipo $B23$ ovvero elementi alla Eulero-Bernoulli specifici per travi snelle e aventi solamente deformazione assiale e a flessione.

Il nodo $3$ è stato scomposto in due diversi nodi sovrapposti $30$ e $31$, a cui tramite la keyword \lstinline{*EQUATION} è stato imposto che lo spostamento orizzontale nel riferimento globale sia uguale per le due aste.

Si riporta ora l'input utilizzato, mentre nelle tabella \ref{tab:SforziAbaqus} e \ref{tab:SpostamentiUabaqus} sono riportati gli sforzi e gli spostamenti ottenuti.
%Input abaqus
\begin{lstlisting} 
*HEADING
ESERCITAZIONE DI MECCANICA COMPUTAZIONALE
Unita di misura SI (Pa, N, m, ...)
1-asse orizzontale, 2-asse verticale
*PREPRINT, ECHO=YES, MODEL=YES, HISTORY=YES

*NODE, NSET=NALL
10, 0, 0, 0
20, 3, 0, 0
30, 3, 3, 0
31, 3, 3, 0
40, 6, 3, 0
50, 6, 1.5, 0
*ELEMENT, TYPE=B23, ELSET=alfazero
2, 20, 30
3, 31, 40
4, 40, 50
5, 50, 20
*ELEMENT, TYPE=B23, ELSET=alfavero
1, 10, 20
*ELEMENT, TYPE=SPRING1, ELSET=MOLLA
6,30
*SPRING, ELSET=MOLLA
1
22289.6
*RELEASE
5,S2,ALLM
*EQUATION 
2 
30, 1, 1.0, 31, 1, -1.0
*INITIAL CONDITIONS, TYPE=TEMPERATURE
NALL,0
*BEAM GENERAL SECTION, ELSET=alfavero, SECTION=GENERAL
0.0019,0.0000028658,0, 0.0000028658,0
0,0,-1
210000000000,8076923077,0.000012
*BEAM GENERAL SECTION, ELSET=alfazero, SECTION=GENERAL
0.0019,0.0000028658,0, 0.0000028658,0
0,0,-1
210000000000,8076923077,0
*STEP, PERTURBATION
*STATIC
*BOUNDARY
10,ENCASTRE
*CLOAD
50,2,-600
*DLOAD
3,P2,-300
*TEMPERATURE, OP=MOD
10,0,-800,0
20,0,-800,0
*NODE PRINT
U,
RF,
*EL PRINT, POSITION=NODES, SUMMARY=YES
SF,
SM,
*END STEP
\end{lstlisting}
%
\begin{table}[htb]
    \centering
    \caption{Risultati degli sforzi ottenuti tramite analisi via FEM}
    \label{tab:SforziAbaqus}
    \begin{tabular}{c
                    c
                    S[table-format=-3.3,
                      table-figures-exponent=2,
                      table-sign-mantissa,
                      table-sign-exponent]    
                    S[table-format=-3.3,
                      table-figures-exponent=2,
                      table-sign-mantissa,
                      table-sign-exponent]}
        \toprule
    	\text{Elemento}&\text{Nodo} & \multicolumn{1}{c}{SF1} & \multicolumn{1}{c}{SM1}\\
    	& & \multicolumn{1}{c}{\SI{}{\newton}} & \multicolumn{1}{c}{\SI{}{\newton\meter}}\\
    	\midrule
        1 & 10 & -96.07     & 7.362e3\\
        1 & 20 & -96.07     & 2.862e3\\
        2 & 20 & 0.000       & 2.386e3\\
        2 & 30 & 0.000       & -6.992E-12\\
        3 & 31 & 1.050e3    & -225.0\\
        3 & 40 & 1.050e3    & 1.125e3\\
        4 & 40 & -900.0     & 1.350e3\\
        4 & 50 & -900.0     & 2.925e3\\
        5 & 50 & -1.610e3   & 2.925e3\\
        5 & 20 & -1,610e3   & -5.002E-12\\
        \bottomrule
    \end{tabular}
\end{table}
%Spostamenti
\begin{table}[htb]
    \centering
    \caption{Risultati degli spostamenti ottenuti tramite analisi via FEM}
    \label{tab:SpostamentiUabaqus}
    \begin{tabular}{c
                    S[table-format=1.2,
                      table-figures-exponent=2,
                      table-sign-mantissa,
                      table-sign-exponent]    
                    S[table-format=1.2,
                      table-figures-exponent=2,
                      table-sign-mantissa,
                      table-sign-exponent]
                    S[table-format=1.2,
                      table-figures-exponent=2,
                      table-sign-mantissa,
                      table-sign-exponent]}  
        \toprule
    	\text{Nodo} & \multicolumn{1}{c}{$U_1$} & \multicolumn{1}{c}{$U_2$}& \multicolumn{1}{c}{$UR_3$}\\
    	& \multicolumn{1}{c}{\SI{}{\meter}} & \multicolumn{1}{c}{\SI{}{\meter}}& \multicolumn{1}{c}{\SI{}{\radian}}\\
    	\midrule
        20 & -7.22E-07 & -6.31E-04 & 3.32E-03    \\
        30 & 4.31E-03  & -6.31E-04 & -3.81E-03   \\
        31 & 4.31E-03  & -3.80E-02 & 7.35E-03    \\
        40 & 4.32E-03  & -1.76E-02 & 5.11E-03    \\
        50 & 8.47E-03  & -1.76E-02 & -2.22E-04   \\
        \bottomrule
    \end{tabular}
\end{table}